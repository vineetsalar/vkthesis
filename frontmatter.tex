%% Title
\titlepage[of Nuclear Physics Division]%
{A dissertation submitted to the HBNI\\
  for the degree of Doctor of Philosophy}

    \begin{figure}
     \includegraphics[width=\smallfigwidth]{HBNI}
  \caption[]{}
  %\label{fig:HBNI}
\end{figure}


%% Abstract
\begin{abstract}%[\smaller \thetitle\\ \vspace*{1cm} \smaller {\theauthor}]

Quarkonia suppression, caused by the Debye screening of the Quantum Chromo
Dynamic (QCD) potential between the two heavy quarks, was originally claimed to
be an unambiguous signal of the formation of a Quark-Gluon Plasma (QGP) \cite{SATZ}. 
J$/\psi$ supression has been observed in heavy ion collisions at both SPS and RHIC, however,
the magnitude of the suppression in both systems was similar despite the differing
energy densities \cite{NA50}. This indicates that Quarkonia suppression isn't as unambiguous
a signal as previously supposed because there are many competing processes that
affect the yield of quarkonia in heavy ion collisions. Some of these effects enhance
the quarkonia yield, such as the statistical recombination of the Q$\overline{Q}$ pairs, while
others reduce it, such as co-mover absorption \cite{Lin}. These processes affect the $\Upsilon$ 
family less than the J$/\psi$  family at 2.76 TeV, and it is believed that color screening should
be the dominant process that contributes to any observed supression of the $\Upsilon$ family. 
%\thispagestyle{empty}
\end{abstract}


%% Declaration
\begin{declaration}
  This dissertation is the result of my own work, except where explicit
  reference is made to the work of others, and has not been submitted
  for another qualification to this or any other university. This
  dissertation does not exceed the word limit for the respective Degree
  Committee.
  \vspace*{1cm}
  \begin{flushright}
    Vineet Kumar
  \end{flushright}
\end{declaration}


%% Acknowledgements
\begin{acknowledgements}
  Of the many people who deserve thanks, some are particularly prominent,
  such as my supervisor
\end{acknowledgements}


%% Preface
\begin{preface}
  This thesis describes my research on various aspects of the CMS
  Heavy Ion physics program, centred around the CMS detector and \LHC
  accelerator at \CERN in Geneva.
  \noindent
  %For this example, I'll just mention \ChapterRef{chap:SomeStuff}
  %and \ChapterRef{chap:MoreStuff}.
\end{preface}

%% ToC
\tableofcontents

%% Strictly optional!
\frontquote%
  {Writing in English is the most ingenious torture\\
   ever devised for sins committed in previous lives.}%
  {James Joyce}

  
