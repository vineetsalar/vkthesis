\graphicspath{{./2fg/}}
\chapter{The LHC and CMS experiment}
\label{ch:LHC}




%%%%%%%%%%%%%%%%%%%%%%%%%%%%%%%%%%%%%%%%%%%%%%%%%%%%%%%%%%%%%%%%%%%%%%%%%%%%%
% PromptChic_CMS_2012.pdf

\section{LHC}
 The Large Hadron Collider (LHC) [2] is the world largest and most powerful particle accelerator.
It is designed to collide proton beams at center of mass energy of 14 TeV. It has
circumference of 27 kms and is placed in a tunnel, 175 meters under the ground near Geneva.
The LHC is the final stage of a system of accelerators shown in Figure 3.1.
Protons in the beams are taken from a bottle of hydrogen gas and first accelerated in linac and
Proton Synchrotron to 26 GeV. Then the particles are injected into Super Proton Synchrotron
and accelerated to 450 GeV. The final acceleration to 7 TeV per proton beam is done in the
two rings of the LHC. There are dipole magnets along the rings which bend the beams. Then
the two beams are focused and brought into collision at four interaction points along the
rings. The proton beams are accelerated with a radio frequency of 400 MHz. This gives rise
to synchrotron oscillations which group the protons in the beams into packets. The LHC is
designed for 2808 packets in a single beam with 25 ns separation and 1.15$ \, \times \,10^{11}$ proton per bunch..

% PromptChic_CMS_2012.pdf ends

% thesisDongho
It is expected to address some of the most
fundamental questions of physics, advancing the understanding of the deepest
laws of nature. The LHC project was approved by the CERN Council in
December 1994. It is originally designed to provide proton-proton collisions
with unprecedented luminosity $L = 10^{34} cm^{-2}s^{-1}$ and a centre-of-mass energy
of 14 TeV. Hadron colliders are well suited to the task of exploring new energy
domains, and the region of 1 TeV constituent centre-of-mass energy can
be explored if the proton energy and the luminosity are high enough. This
is because a hadron machine thanks to the lower amount of synchrotron radiation
emitted by circulating hadrons, which is proportional to the fourth
power of {\bf E/m}. Compared to electrons, the energy loss is reduced by a factor of
$\langle \mathcal{O}^{12}\rangle$. In addition to $p + p$ operation, the LHC had 
heavy nuclei $(Pb + Pb)$ collisions in 2009 and 2011 
 with an energy of 2.76 TeV per nucleon. The availability of high energy heavy-ion beams at energies
over 30 times higher than at the present other accelerators will allow us to
further extend the range of the heavy-ion physics program to include studies
of hot nuclear matter.

%\section{Lattice layout}
The two LHC symmetrical rings are divided into eight octants and arcs and
eight straight sections approximately 528 $m$ (Fig. 3.4). The two high luminosity
experimental insertions are located at diametrically opposite straight
sections: the A Toroidal LHC ApparatuS (ATLAS Fig. 3.5) experiment is located
at Point 1 and the Compact Muon Solenoid (CMS Fig. 3.6) experiment
at Point 5. The other two large experiments, A Large Ion Collider Experiment
(ALICE Fig. 3.7) and Large Hadron Collider beauty (LHCb 3.8), are located
at Point 2 and at Point 8, respectively, where the machine reaches a lower
luminosity of $L = 5 \, \times \,10^{32} cm^{-2}s^{-1}$. The remaining four straight sections do
not have beam crossings. The two beams are injected into the LHC in two
different octants, octant 2 and octant 8 respectively for clockwise and anticlockwise
beam. The octants 3 and 7, instead, contain two collimation systems
for the beam cleaning. 

\section{Luminosity}
The number of events per second generated in the LHC collisions is given by :
N = $L.\sigma$
where $\sigma$ is the cross section for the collisions process under study and $L$ the
machine luminosity. The machine luminosity depends only on the beam parameters and can be written, 
for a Gaussian beam distribution, as:

% thesisDongho ends

% PromptChic_CMS_2012.pdf 

\begin{equation}
\label{eq:Lumi}
L = \frac{n \cdot \,f_{rev}\cdot \,N_{1} \cdot \, N_{2}}{A^{eff}_T}
\end{equation}

where $A^{eff}_T$ is the effective transverse area of the proton beam, $n$ is the number of packets
the beam is splitted to and $f_{rev}$ is the frequency of revolution around the ring. $N_{1}$ and $N_{2}$ are
the number of protons in each packet. With respect to other high energy colliders, the design
luminosity of LHC is several magnitudes larger. This is needed because LHC is designed
to discover new particles at TeV scale. At these scales the interaction rates with momentum
transfers more than 1 TeV are very low. Therefore more data needs to be collected which can
only be achieved by having large luminosity.

% PromptChic_CMS_2012.pdf ends

% thesisDongho 
The LHC luminosity is not constant over physics a run, but decays due to the
degradation of intensities and emittance of circulating beams. The main cause
of the luminosity decay during normal LHC operation is the beam loss from
collisions.

\section{Compact Muon Solenoid}

The Compact Muon Solenoid (CMS) (see Fig. 3.11) is a high granularity detector
built around and inside a superconducting solenoid that provides a strong
magnetic field of 3.8 $T$. Inside it, the inner tracking comprises a Pixel detector
surrounded by the Silicon Strip detector. Its high granularity (70
millions pixels, 10 millions strip) and precision ensures good track reconstruction
efficiency. It is surrounded by Electromagnetic calorimeter (ECAL) made
of 76000 lead tungstate crystals grouped in 36 barrel and 4 endcap supermodules.
The brass-scintillator sampling hadron calorimeter (HCAL) completes
the in-coil detectors. To ensure hermeticity the in-coil calorimetric system is
extended, away from the central dector, by the hadron outer detector (HO)
and a quartz fiber very forward calorimeter (HF) to cover $\eta le $< 5. Outside the
solenoid a muon system is built in the magnet steel return yoke. It's formed by
4 stations of muon chambers: Drift Tube (DT) in the barrel region, Cathode
Strip Chambers (CSC) in the endcap, Resistive Plate Chamber (RPC) in both
parts, providing muon detection redundancy. Two trigger levels are employed
in CMS. The Level-1 Trigger (L1) is implemented using custom hardware processors
and is designed to reduce the event rate to 100 kHz during LHC operation
using information from the calorimeters and the muon detectors. It operates nearly 
dead time-free and synchronously with the LHC bunch crossing
frequency of 40 MHz. The High Level Trigger (HLT) is implemented across a
large cluster of commodity computers referred to as the event filter farm, and
provides further rate reduction to $\mathcal{O}$(100) Hz using filtering software applied
to data from all detectors at full granularity. The overall dimension of CMS
are a length of 21.6 $m$, a diameter of 14.6 $m$ and a total weight of 12500 tons.
In the CMS collaboration is adopted the following system of coordinates.


\subsection{Compact Muon Solenoid}

The coordinate system of CMS has its origin inside the detector at the primary
interaction point. The x-axis points radially towards the center of the
LHC, whereas the y-axis points vertically upward. Thus, the z-axis shares the
same direction with the beam line. The azimuthal angle $\phi$ is measured from
the x-axis in the xy plane whereas the polar angle $\theta$ is measured from the z-axis.
Particle physicists often use a Lorentz invariant quantity called rapidity
$y$ instead of $\theta$. It is defined as:

\begin{equation}
\label{eq:CMSqd1}
 y = \frac{1}{2} \,\ln \frac{E + p_z}{E - p_z} \, = tanh^{-1} \,\frac{p_z}{E}
\end{equation}

and equals, in case of massless particles, the pseudorapidity $\eta $ given by

\begin{equation}
\label{eq:CMSqd2}
 y = -\ln[\tan(\theta/2)]
\end{equation}

The angular distance between two particles observed from the origin of the
coordinate system is

\begin{equation}
\label{eq:CMSqd3}
\Delta R = \sqrt{\Delta \phi^2 + \Delta \eta^2} 
\end{equation}

Measurable quantities like momentum and energy transverse to the beam line
are denoted by $p_T$ and $E_T$ , respectively, and can be derived from its x and
y components. The CMS detector is located north of the LHC center. The
origin of the CMS coordinate system is the CMS collision point. Neglecting
the small tilt of the LEP/LHC plane, the coordinate system is as follows:
